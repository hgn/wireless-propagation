\documentclass[9pt]{article}
 
\usepackage{times}
\usepackage{transparency}
\usepackage[german]{babel}
\usepackage[T1]{fontenc}
\usepackage[latin1]{inputenc}
\usepackage{subfigure}

\screensize{8.5truein}{11truein}
\begin{document}


\title{On the Validation of Radio Propagation Models}

\setbackground{background.png}
\subtitle{Network Simulator driven Propagation Behavior}

\name{Hagen Paul Pfeifer}
\footerline{Radio Propagation Models}
\email{hagen.pfeifer@protocollabs.de}
\company{ProtocolLabs\\http://www.protocollabs.com\\Munich, Germany}
\date{\today}
\maketitle

%%%%%%%%%%%%%%%%%%%%%%%%%%%%%%
\begin{slide}
\slidetitle{Disclaimer}{Disclaimer}
\bi
	\item The initial purpose of this work was to verify and
	      check the implementation of the \textit{Nakagami Fading Model}
		  in ns-3. In the half of the work I realized that the generated
		  material is a good starting point for path loss in general.
		  Especially to develop intuition how path loss influence the Wireless
		  simulation at the whole. How model knobs influences the
		  characteristic of the channel, etc.

	\item The complete work is public available and can be used for further
	      investigation:\newline
	      \verb+git clone http://git.jauu.net/wireless-propagation.git+

	\item Suggestion, enhancement or critic is highly welcome!
\ei
\end{slide}

%%%%%%%%%%%%%%%%%%%%%%%%%%%%%%
\begin{slide}
\slidetitle{Agenda}{Agenda}
\bi
	\item Question: For a given distance between two Wireless Stations
	      how to determine the propability if a transmitted packet is
		  receptable?\\
		  Or in other words: how to simulate the physical radio channel?
\ei
\end{slide}


%%%%%%%%%%%%%%%%%%%%%%%%%%%%%%
\begin{slide}
\slidetitle{Theoretical Background}{Theoretical Background}
\bi
	\item A Wireless channel is unsteady and lossy

	\item Wireless network simulators requires a model of this characteristic

	\item Three parameter are most relevant:

	\be
		\item Attenuation
		\item Slow Fading (shadowing)
		\item Fast Fading (multipath scattering)
	\ee
\ei
\end{slide}

%%%%%%%%%%%%%%%%%%%%%%%%%%%%%%
\begin{slide}
\slidetitle{Fresnel Zone}{Fresnel Zone}
\begin{picture}(0,0)
\put(100,-500){\includegraphics[scale=0.7]{images/fresnel_zone.png}}
\end{picture}

\bi
	\item How to find the right place for a radio equipment?
	\item In order for radio waves emitted from the transmitter
	      to reach the receiver without attenuation of power, a
		  certain amount of space is required
	\item The space required is a spheroid with its center along the
		  shortest distance between antennas, and this is called the Fresnel zone
\ei
\end{slide}

%%%%%%%%%%%%%%%%%%%%%%%%%%%%%%
\begin{slide}
\slidetitle{Fresnel Zone}{Fresnel Zone}
\bi
	\item The 1st Fresnel (pronounced Fray-nell) zone is a spheroid
		  space formed within the trajectory of the path when the path difference
		  when radio wave energy reaches the receiver by the shortest distance, and
		  when it gets there by another route, is within λ/2. In this case, λ is the
		  wave length of the radio wave (wave length = speed of light / frequency)
		  which at 400 MHz is 0.75 m.
	\item if there are no obstacles in the space forming 60\%
	      of this distance, propagation characteristics are said to be
		  the same as in free space
	\item ensure "line of sight" between the transceivers  - line of sight for
		  radio wave betrifft die frezel zone!
	\item if a Fresnel zone is not established, multipath interference will occur
	\item The Frezelzone is Frequenzabhaengig: >> F -> <b
	\item r = 17.31 * sqrt(N(d1*d2)/(f*d))
	\item r = 17.31 sqrt(1 * (1000 * 1000) / (2437 * 2000))
	\item r = 7.84 meters
\ei
\end{slide}

%%%%%%%%%%%%%%%%%%%%%%%%%%%%%%
\begin{slide}
\slidetitle{Common dbM/Watt Values}{Common dbM/Watt Values\footnote{Source: WP}}
\bi
	\item 80 dBm	100 kW	Typical transmission power of FM radio station with 50 km range
	\item 60 dBm	1 kW = 1000 W	Typical combined radiated RF power of microwave oven elements
	\item 33 dBm	2 W		Maximum output from a UMTS/3G mobile phone (Power class 1 mobiles)
	\item 30 dBm	1 W = 1000 mW	Typical RF leakage from a microwave oven
	\item 20 dBm	100 mW	Bluetooth Class 1 radio, 100 m range
	\item 15 dBm	32 mW	Typical WiFi transmission power in laptops
	\item 4 dBm		2.5 mW	Bluetooth Class 2 radio, 10 m range
	\item 0 dBm		1.0 mW = 1000 $mu$W	Bluetooth standard (Class 3) radio, 1 m range
	\item -10 dBm	100 $mu$W	Typical maximum received signal power (-10 to -30 dBm) of wireless network
	\item -70 dBm	100 pW	Typical range (-60 to -80 dBm) of wireless (802.11x) received signal power over a network
	\item -127.5 dBm	0.178 fW = 178 aW	Typical received signal power from a GPS satellite
	\item -174 dBm	0.004 aW = 4 zW		Thermal noise floor for 1 Hz bandwidth at room temperature (20 C)
\ei
\end{slide}

%%%%%%%%%%%%%%%%%%%%%%%%%%%%%%
\begin{slide}
\slidetitle{}{}
\bi
	\item 
\ei
\end{slide}

%%%%%%%%%%%%%%%%%%%%%%%%%%%%%%
\begin{slide}
\slidetitle{Friis}{Friis}
\bi
	\item Formular:\\\vspace{1cm}
	\begin{large}
	\begin{math}
	\frac{P_r}{P_t} = G_t G_r (\frac{\lambda}{4 \pi R} )^2
	\end{math}\\\vspace{1cm}
	\end{large}
	\begin{small}
	$P_r$ Reveiving power (dBm)\\
	$P_t$ Transmitter power (dBm)\\
	$G_t$ Antenna Gain Transmitter (dBi/dBd)\\
	$G_r$ Antenna Gain Receiver (dBi/dBd)\\
	$\lambda$ Wavelength (meter, \dots)\\
	$R$ Distance between the nodes (meter, \dots)\\
	\end{small}
\ei
\end{slide}

%%%%%%%%%%%%%%%%%%%%%%%%%%%%%%
\begin{slide}
		\begin{picture}(0,0)
		\put(100,-400){\includegraphics[scale=0.7]{images/friis.pdf}}
		\end{picture}
\slidetitle{Friis}{Friis}
\end{slide}

%%%%%%%%%%%%%%%%%%%%%%%%%%%%%%
\begin{slide}
\slidetitle{Friis Wavelength Influences}{Friis Wavelength Influences}
			\begin{figure}[ht]
			\centering
			\subfigure[\tiny{Frequency: 900 Mhz (GSM)}]{\includegraphics[scale=0.45]{images/friis900.pdf}}
			\subfigure[\tiny{Frequency: 5.0 GHz (802.11)}]{\includegraphics[scale=0.45]{images/friis5000.pdf}}
			\end{figure}
\bi
	\item The higher the frquency, the higher the loss.
\ei
\end{slide}

%%%%%%%%%%%%%%%%%%%%%%%%%%%%%%
\begin{slide}
		\begin{picture}(0,0)
		\put(100,-400){\includegraphics[scale=0.7]{images/tworayground.pdf}}
		\end{picture}
\slidetitle{Two Ray Ground (vanilla)}{Two Ray Ground (vanilla)}
\end{slide}

%%%%%%%%%%%%%%%%%%%%%%%%%%%%%%
\begin{slide}
		\begin{picture}(0,0)
		\put(100,-400){\includegraphics[scale=0.7]{images/tworaygroundvanilla.pdf}}
		\end{picture}
\slidetitle{Two Ray Ground}{Two Ray Ground}
\end{slide}

%%%%%%%%%%%%%%%%%%%%%%%%%%%%%%
\begin{slide}
		\begin{picture}(0,0)
		\put(100,-400){\includegraphics[scale=0.7]{images/shadowing.pdf}}
		\end{picture}
\slidetitle{Shadowing Model}{Shadowing Model}
\end{slide}

%%%%%%%%%%%%%%%%%%%%%%%%%%%%%%
\begin{slide}
		\begin{picture}(0,0)
		\put(100,-400){\includegraphics[scale=0.7]{images/nakagami.pdf}}
		\end{picture}
\slidetitle{Nakagami Model}{Nakagami Model}
\end{slide}

%%%%%%%%%%%%%%%%%%%%%%%%%%%%%%
\begin{slide}
		\begin{picture}(0,0)
		\put(100,-400){\includegraphics[scale=0.7]{images/nakagami_distribution.pdf}}
		\end{picture}
\slidetitle{Nakagami Model Distribution}{Nakagami Model Distribution}
\end{slide}

%%%%%%%%%%%%%%%%%%%%%%%%%%%%%%
\begin{slide}
		\begin{picture}(0,0)
		\put(100,-400){\includegraphics[scale=0.7]{images/nakagami_m0_variances.pdf}}
		\end{picture}
\slidetitle{Nakagami Model m0 Effect}{Nakagami Model m0 Effect}
\end{slide}

%%%%%%%%%%%%%%%%%%%%%%%%%%%%%%
\begin{slide}
		\begin{picture}(0,0)
		\put(100,-400){\includegraphics[scale=0.7]{images/nakagami_m0_variance_distribution.pdf}}
		\end{picture}
\slidetitle{Nakagami Model m0 Effect}{Nakagami Model m0 Effect}
\end{slide}

%%%%%%%%%%%%%%%%%%%%%%%%%%%%%%
\begin{slide}
		\begin{picture}(0,0)
		\put(100,-400){\includegraphics[scale=0.7]{images/logdistance.pdf}}
		\end{picture}
\slidetitle{Log Distance Model}{Log Distance Model}
\end{slide}

%%%%%%%%%%%%%%%%%%%%%%%%%%%%%%
\begin{slide}
\begin{picture}(0,0)
\put(100,-400){\includegraphics[scale=0.7]{images/threelogdistance.pdf}}
\end{picture}
\slidetitle{Three Log Distance Model}{Three Log Distance Model}
\end{slide}

%%%%%%%%%%%%%%%%%%%%%%%%%%%%%%
\begin{slide}
\slidetitle{TX Power to SNR}{TX Power to SNR}
\bi
	\item Signal-to-Noise Ratio (SNR or S/N)
	\item $SNR = \frac{P_{signal}}{P_{noise}}$
	\item Noise:
	\bi
		\item Boltzmann constant * Bandwidth * Receiver Noise * Implementation Loss
		\item Boltzmann constant ($k_B$): $3.91^{-21}$ (B * Temp in Kelvin)
		\item Bandwidth: $20^6$ Hz
		\item Receiver Noise: 15.8 W (~12 dB)
		\item Implementation Loss: 1.58 W (2 dB)
	\ei
	\item Noise: $1.99^{-12}$ Watt = $-87 dBm$
	\item Example:
	\bi
		\item Signal: -60 dBm
		\item Noise: -87 dBm
		\item SNR: -27 dB
	\ei
\ei
\end{slide}


%%%%%%%%%%%%%%%%%%%%%%%%%%%%%%
\begin{slide}
\slidetitle{Symbol Error Rate}{Symbol Error Rates}
\bi
	\item Symbol Error Rate
	\item common digital modulation schemes:
	\bi
		\item BPSK (2 Symbols)
		\item QPSK (4 Symbols)
		\item 8PSK (8 Symbols)
		\item QAM (16, 32, 64 Symbols)
	\ei
\ei
\end{slide}


%%%%%%%%%%%%%%%%%%%%%%%%%%%%%%
\begin{slide}
\slidetitle{Bit Error Rate}{Bit Error Rate}
\begin{picture}(0,0)
\put(100,-400){\includegraphics[scale=.7]{images/formel.png}}
\end{picture}
\begin{picture}(0,0)
\put(0,-200){\includegraphics[scale=.7]{images/srr.png}}
\end{picture}
\bi
	\item Symbol Error Rate vs. Es/No
	\item Symbol Error Rate $\rightarrow$ Bit Error Rate
	\item $\frac{P_{S,MQAM}}{Bits/Symbol}$
\ei
\end{slide}

%%%%%%%%%%%%%%%%%%%%%%%%%%%%%%
\begin{slide}
\slidetitle{The End}{The End}
\bi
	\item Questions?
\ei
\end{slide}

%%%%%%%%%%%%%%%%%%%%%%%%%%%%%%
\begin{slide}
\slidetitle{SINR and IPW2200}{SINR and IPW2200}
\bi
	\item Signal-To-Noise Ratio (aka SNR and S/R)
	\item snr(dB) = signal level(dBm) - average noise level(dBm)
	\item Ratio of a signal power to the noise power corrupting the signal
	\item To receive a useful information the signal must clearly be higher then the noise
	\item Sidenote IPW2200 driver and \verb+/proc/net/wireless+
	\bi
		\item \verb+iwconfig eth1+ $\rightarrow$ Link Quality=69/100  Signal level=-55 dBm  Noise level=-82 dBm
		\item Noise: initial set to -85 dBm
		\item Every new packet updates the noise level (and average it with the previous one)
		\item \verb+priv->exp_avg_noise+
		\item RSSI: received signal strength indication
		\item RSSI: a measurement of the power present in a received radio signal
		\item 0 to 255
		\item RSSI is acquired during the preamble stage of receiving an 802.11 frame
		\item RSSI is stored on the RX descriptor (stats.rssi) and is measured by baseband and PHY for each individual packet
		\item see \verb+drivers/net/wireless/ipw2x00/ipw2200.c:ipw_rx()+
	\ei
\ei
\end{slide}

%%%%%%%%%%%%%%%%%%%%%%%%%%%%%%
\begin{slide}
\slidetitle{Some Background on Antennas}{Some Background on Antennas}
\begin{picture}(0,0)
\put(50,-200){\includegraphics[scale=0.25]{images/polar_pattern_directional.png}}
\end{picture}
\begin{picture}(0,0)
\put(50,-450){\includegraphics[scale=0.25]{images/polar_pattern_omnidirectional.png}}
\end{picture}
\bi
	\item Basic types
	\bi
		\item Omnidirectional
		\item Semi-directional
		\item Directional
	\ei
	\item Omnidirectional
	\bi
		\item Omnidirectional antennas radiate energy equally in  all directions around the antenna's vertical axis
		\item Most common for WLAN: dipole antenna
	\ei
	\item Semi-Directional
	\bi
		\item Patch
		\item Panel
		\item Yagi
		\item Common examples: TV antennas or Cellular repeaters antennas
	\ei
	\item Highly Directional
	\bi
		\item Parabolic dish
		\item Grid antenna
	\ei
\ei
\end{slide}

\end{document}

% vim600: fdm=marker tw=130 sw=4 ts=4 sts=4 ff=unix noet:
