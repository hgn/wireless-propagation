\documentclass[9pt]{article}
 
\usepackage{times}
\usepackage{transparency}
\usepackage[german]{babel}
\usepackage[T1]{fontenc}
\usepackage[latin1]{inputenc}

\screensize{8.5truein}{11truein}
\begin{document}


\title{Radio Propagation Models}

\setbackground{background.png}
\subtitle{}

\name{Hagen Paul Pfeifer}
\footerline{Radio Propagation Models}
\email{hagen.pfeifer@protocollabs.de}
\company{}
\date{\today}
\maketitle

%%%%%%%%%%%%%%%%%%%%%%%%%%%%%%
\begin{slide}
\slidetitle{Disclaimer}{Disclaimer}
\bi
	\item The initial purpose of this work was to verify and
	      check the implementation of the \textit{Nakagami Fading Model}
		  in ns-3. In the half of the work I realized that the generated
		  material is a good starting point for path loss in general.
		  Especially to develop intuition how path loss influence the Wireless
		  simulation at the whole. How model knobs influences the
		  characteristic of the channel, etc.

	\item The complete work is public available and can be used for further
	      investigation:\newline
	      \verb+git clone git.jauu.net/rpm-analysis.git+

	\item Suggestions, enhancements or critic is highly welcome!
\ei
\end{slide}

%%%%%%%%%%%%%%%%%%%%%%%%%%%%%%
\begin{slide}
\slidetitle{Theoretical Background}{Theoretical Background}
\bi
	\item A Wireless channel is unsteady and lossy

	\item Wireless network simulators requires a model of this characteristic

	\item Three parameter are most relevant:

	\be
		\item Attenuation
		\item Slow Fading (shadowing)
		\item Fast Fading (multipath scattering)
	\ee
\ei
\end{slide}


%%%%%%%%%%%%%%%%%%%%%%%%%%%%%%
\begin{slide}
\slidetitle{Friis}{Friis}
\bi
	\item Formular:\\\vspace{1cm}
	\begin{large}
	\begin{math}
	\frac{P_r}{P_t} = G_t G_r (\frac{\lambda}{4 \pi R} )^2
	\end{math}\\\vspace{1cm}
	\end{large}
	\begin{small}
	$P_r$ Reveiving power (dBm)\\
	$P_t$ Transmitter power (dBm)\\
	$G_t$ Antenna Gain Transmitter (dBi/dBd)\\
	$G_r$ Antenna Gain Receiver (dBi/dBd)\\
	$\lambda$ Wavelength (meter, \dots)\\
	$R$ Distance between the nodes (meter, \dots)\\
	\end{small}
\ei
\end{slide}

%%%%%%%%%%%%%%%%%%%%%%%%%%%%%%
\begin{slide}
		\begin{picture}(0,0)
		\put(100,-400){\includegraphics[scale=0.7]{images/friis.pdf}}
		\end{picture}
\slidetitle{Friis}{Friis}
\end{slide}

%%%%%%%%%%%%%%%%%%%%%%%%%%%%%%
\begin{slide}
		\begin{picture}(0,0)
		\put(100,-400){\includegraphics[scale=0.7]{images/tworayground.pdf}}
		\end{picture}
\slidetitle{Two Ray Ground (vanilla)}{Two Ray Ground (vanilla)}
\end{slide}

%%%%%%%%%%%%%%%%%%%%%%%%%%%%%%
\begin{slide}
		\begin{picture}(0,0)
		\put(100,-400){\includegraphics[scale=0.7]{images/tworaygroundvanilla.pdf}}
		\end{picture}
\slidetitle{Two Ray Ground}{Two Ray Ground}
\end{slide}

%%%%%%%%%%%%%%%%%%%%%%%%%%%%%%
\begin{slide}
		\begin{picture}(0,0)
		\put(100,-400){\includegraphics[scale=0.7]{images/shadowing.pdf}}
		\end{picture}
\slidetitle{Shadowing Model}{Shadowing Model}
\end{slide}

%%%%%%%%%%%%%%%%%%%%%%%%%%%%%%
\begin{slide}
		\begin{picture}(0,0)
		\put(100,-400){\includegraphics[scale=0.7]{images/nakagami.pdf}}
		\end{picture}
\slidetitle{Nakagami Model}{Nakagami Model}
\end{slide}

%%%%%%%%%%%%%%%%%%%%%%%%%%%%%%
\begin{slide}
		\begin{picture}(0,0)
		\put(100,-400){\includegraphics[scale=0.7]{images/logdistance.pdf}}
		\end{picture}
\slidetitle{Log Distance Model}{Log Distance Model}
\end{slide}

%%%%%%%%%%%%%%%%%%%%%%%%%%%%%%
\begin{slide}
		\begin{picture}(0,0)
		\put(100,-400){\includegraphics[scale=0.7]{images/threelogdistance.pdf}}
		\end{picture}
\slidetitle{Three Log Distance Model}{Three Log Distance Model}
\end{slide}



\end{document}

% vim600: fdm=marker tw=130 sw=4 ts=4 sts=4 ff=unix noet:
